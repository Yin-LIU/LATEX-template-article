\documentclass[usletter,12pt]{article}
\usepackage{basic_template}
% Math symbols and related macros
\newcommand{\grad}{\nabla}
\newcommand{\noi}{\noindent}

% Boldface vectors and matrices
\def\ba{\mathbf{a}}
\def\bb{\mathbf{b}}
\def\bc{\mathbf{c}}
\def\bd{\mathbf{d}}
\def\be{\mathbf{e}}
\def\bff{\mathbf{f}}
\def\bg{\mathbf{g}}
\def\bh{\mathbf{h}}
\def\bi{\mathbf{i}}
\def\bj{\mathbf{j}}
\def\bk{\mathbf{k}}
\def\bl{\mathbf{l}}
\def\bm{\mathbf{m}}
\def\bn{\mathbf{n}}
\def\bp{\mathbf{p}}
\def\bq{\mathbf{q}}
\def\br{\mathbf{r}}
\def\bs{\mathbf{s}}
\def\bt{\mathbf{t}}
\def\bu{\mathbf{u}}
\def\bv{\mathbf{v}}
\def\bw{\mathbf{w}}
\def\bx{\mathbf{x}}  %{\mbox{\boldmath $\lambda$}}
\def\by{\mathbf{y}}
\def\bz{\mathbf{z}}

\def\bA{\mathbf{A}}
\def\bB{\mathbf{B}}
\def\bC{\mathbf{C}}
\def\bD{\mathbf{D}}
\def\bE{\mathbf{E}}
\def\bF{\mathbf{F}}
\def\bG{\mathbf{G}}
\def\bH{\mathbf{H}}
\def\bI{\mathbf{I}}
\def\bJ{\mathbf{J}}
\def\bK{\mathbf{K}}
\def\bL{\mathbf{L}}
\def\bM{\mathbf{M}}
\def\bN{\mathbf{N}}
\def\bO{\mathbf{O}}
\def\bP{\mathbf{P}}
\def\bQ{\mathbf{Q}}
\def\bR{\mathbf{R}}
\def\bS{\mathbf{S}}
\def\bU{\mathbf{U}}
\def\bT{\mathbf{T}}
\def\bY{\mathbf{Y}}
\def\bX{\mathbf{X}}
\def\bV{\mathbf{V}}
\def\bW{\mathbf{W}}
\def\bZ{\mathbf{Z}}
\def\bo{\mathbf{0}}


% Bold Greek letters
\def\e{{\boldsymbol{\epsilon}}}
\def\l{{\boldsymbol{\lambda}}}
\def\g{{\boldsymbol{\gamma}}}
\def\G{{\boldsymbol{\Gamma}}}
% \def\L{{\boldsymbol{\Lambda}}}
\def\D{{\boldsymbol{\Delta}}}
\def\a{{\boldsymbol{\alpha}}}
\def\m{{\boldsymbol{\mu}}}
\def\n{{\boldsymbol{\nu}}}
%\def\S{{\boldsymbol{\Sigma}}}
\def\p{{\boldsymbol{\pi}}}
\def\th{{\boldsymbol{\theta}}}
\def\Th{{\boldsymbol{\Theta}}}
\def\x{{\boldsymbol{\xi}}}
\def\b{{\boldsymbol{\beta}}}
\def\r{{\boldsymbol{\rho}}}
\def\z{{\boldsymbol{\zeta}}}
\def\ph{{\boldsymbol{\phi}}}
\def\ps{{\boldsymbol{\psi}}}
\def\pis{\pi^{\star}}

% Calligraphic letters
\def\cA{\mathcal{A}}
\def\cB{\mathcal{B}}
\def\cC{\mathcal{C}}
\def\cD{\mathcal{D}}
\def\cE{\mathcal{E}}
\def\cF{\mathcal{F}}
\def\cG{\mathcal{G}}
\def\cH{\mathcal{H}}
\def\cI{\mathcal{I}}
\def\cJ{\mathcal{J}}
\def\cK{\mathcal{K}}
\def\cL{\mathcal{L}}
\def\cM{\mathcal{M}}
\def\cN{\mathcal{N}}
\def\cO{\mathcal{O}}
\def\cP{\mathcal{P}}
\def\cQ{\mathcal{Q}}
\def\cR{\mathcal{R}}
\def\cS{\mathcal{S}}
\def\cT{\mathcal{T}}
\def\cU{\mathcal{U}}
\def\cV{\mathcal{V}}
\def\cW{\mathcal{W}}
\def\cX{\mathcal{X}}
\def\cY{\mathcal{Y}}
\def\cZ{\mathcal{Z}}

% Blackboard bold letters (typically for sets like reals, integers, etc.)
\def\mA{\mathbb{A}}
\def\mB{\mathbb{B}}
\def\mC{\mathbb{C}}
\def\mD{\mathbb{D}}
\def\mE{\mathbb{E}}
\def\mF{\mathbb{F}}
\def\mG{\mathbb{G}}
\def\mH{\mathbb{H}}
\def\mI{\mathbb{I}}
\def\mJ{\mathbb{J}}
\def\mK{\mathbb{K}}
\def\mL{\mathbb{L}}
\def\mM{\mathbb{M}}
\def\mN{\mathbb{N}}
\def\mO{\mathbb{O}}
\def\mP{\mathbb{P}}
\def\mQ{\mathbb{Q}}
\def\mR{\mathbb{R}}
\def\mS{\mathbb{S}}
\def\mT{\mathbb{T}}
\def\mU{\mathbb{U}}
\def\mV{\mathbb{V}}
\def\mW{\mathbb{W}}
\def\mX{\mathbb{X}}
\def\mY{\mathbb{Y}}
\def\mZ{\mathbb{Z}}

% Mathematical constructs
\newcommand{\abs}[1]{\left|#1\right|}
\newcommand{\norm}[1]{\left\|#1\right\|}
%\newcommand{\vec}[1]{\begin{pmatrix}#1\end{pmatrix}}
\newcommand{\bmat}[1]{\begin{bmatrix}#1\end{bmatrix}}

% Abbreviations for list environments
% \newcommand{\BIT}{\begin{itemize}}
% \newcommand{\EIT}{\end{itemize}}
% \newcommand{\BNUM}{\begin{enumerate}}
% \newcommand{\ENUM}{\end{enumerate}}

% Standard math stuff
% \newcommand{\reals}{\mathbb{R}}
% \newcommand{\eqbydef}{\stackrel{\Delta}{=}}

% Operators and other constructs
\DeclareMathOperator*{\supp}{supp}
\DeclareMathOperator*{\argmin}{argmin}
\DeclareMathOperator*{\argmax}{argmax}
\DeclareMathOperator{\Tr}{Tr}
\DeclareMathOperator{\Rank}{rank}
\def\fprod#1{\left\langle#1\right\rangle}
\def\prox#1{\mathbf{prox}_{#1}}

\newcommand{\dist}{\mathop{\bf dist{}}}
\newcommand{\dom}{\mathop{\bf dom}}

\DeclareMathOperator*{\cov}{Cov}
\DeclareMathOperator*{\var}{Var}
\DeclareMathOperator*{\sgn}{sgn}


% Text abbreviations
\newcommand{\cf}{{cf.\ }}
\newcommand{\eg}{{e.g.,\ }}
\newcommand{\ie}{{i.e.,\ }}
\newcommand{\etc}{{etc.\ }}

% colored text
\def\red#1{\textcolor{red}{#1}}
\def\green#1{\textcolor{green}{#1}}
\def\blue#1{\textcolor{blue}{#1}}
\def\tri#1{\textcolor{#1}{$\blacktriangleright$}}
\def\brown#1{\textcolor{brown}{#1}}


\newcommand\numberthis{\addtocounter{equation}{1}\tag{\theequation}}


%%%%%%%%%%%%%%%%%%%%%%%%%%%%%%%%%%%%%%%%%%%%
% Use Natbib for bibliography management with numerical citations
\usepackage[numbers]{natbib}

% Define authors for the 'changes' package with specific colors for each ID
\defineauthor{ID1}{purple}
\defineauthor{ID2}{teal}

% \setuptodonotes{inline} % Uncomment if you want comments to appear inline or in the margin
%%%%%%%%%%%%%%%%%%%%%%%%%%%%%%%%%%%%%%%%%%%%%

% Document title
\title{\textbf{Your Document Title Here}}

% Author and affiliation section with email addresses
\author[ ]{\textbf{Author 1}}
\author[$\ddag$]{\textbf{Author 2}}
\author[*]{\textbf{Author 3}}

\affil[ ]{Department of XXX, University of XXX \protect\\
          \texttt{\{email1, email3\}@xxx.edu}}
\affil[$\ddag$]{Department of YYY, University of YYY \protect\\
          \texttt{email2@yyy.edu}}

\affil[*]{Department of ZZZ, University of YYY \protect\\
          \texttt{email2@yyy.edu}}

% Adjust the spacing between author names
\renewcommand\Authsep{\quad}
\renewcommand\Authands{\quad}

% Set the document's date to the current date
\date{\today}

% Begin the document content
\begin{document}

% Display the title
\maketitle

% Abstract section
\begin{abstract}
    Your abstract text goes here. The abstract provides a concise summary of your paper's goals, methods, results, and conclusions.
\end{abstract}

% Uncomment the following lines if a table of contents is required
%\newpage
%\tableofcontents

\section{Introduction}

This template includes essential packages for paper writing and incorporates ideas from the changes package. It allows authors to be defined with unique IDs and colors, making it easy to identify who is responsible for each modification. When using this package, you can load it with the following command:

\texttt{\textbackslash usepackage\{basic\_template\}}.

If you want to  compile the document with real-time preview in Texifier, use the command 

\texttt{\textbackslash [texifier] usepackage\{basic\_template\}}.

But formatting may be incorrect due to conflicts with Texifier's functionality).

With this package, the following commands can be used to clearly display changes made to the document:
\begin{itemize}
    \item \texttt{\textbackslash added[ID1]\{Text added by ID1\}} \added[ID1]{Text added by ID1}
    \item \texttt{\textbackslash deleted[ID1]\{Text deleted by ID1\}} \deleted[ID1]{Text deleted by ID1}
    \item \texttt{\textbackslash replaced[ID1]\{New text\}\{Old text\}} \replaced[ID1]{New text}{Old text}
    \item \texttt{\textbackslash added[ID2]\{Text added by ID2\}} \added[ID2]{Text added by ID2}
    \item \texttt{\textbackslash deleted[ID2]\{Text deleted by ID2\}} \deleted[ID2]{Text deleted by ID2}
    \item \texttt{\textbackslash replaced[ID2]\{New text\}\{Old text\}} \replaced[ID2]{New text}{Old text}
    \item \texttt{\textbackslash note[ID1]\{Comment by ID1\}} \note[ID1]{Comment by ID1}
    \item If no author ID is assigned, the default color (blue) will be used. \added{The default color for changes is blue.}
\end{itemize}

These commands also work within mathematical expressions and across multiple paragraphs. For example:

\added[ID1]{
This is the first paragraph. 

This is the second paragraph.
\begin{align*}
    f(x) &= \frac{1}{2}x^2 \\
    g(x) &= x^3
\end{align*}
}

\replaced[ID1]{This is the new part}{This is the first deleted paragraph.

This is the second paragraph.
\begin{align*}
    f(x) &= \frac{1}{2}x^2 \\
    g(x) &= x^3
\end{align*}}

For citing references, you can use \citep{Cyrus2018RobustAcceleratedOptimization}.

\newpage
% Bibliography section
\bibliographystyle{plainnat}
\bibliography{refs.bib}

\newpage
% Appendices section
\begin{appendices}
\section{Appendix Section Title}
% Additional appendices content here
\end{appendices}

\end{document}
